\chapter{\label{ch:ch01}ОБЗОР ПРЕДМЕТНОЙ ОБЛАСТИ} % Нужно сделать главу в содержании заглавными буквами

\section{\label{sec:ch01/sec01} Арканоид}

\textbf{Общие сведения об игре "Арканоид"}

Игра Арканоид — классическая аркадная игра, выпущенная в 1986 году компанией Taito.
~\cite{wikiRUArkanoid} Цель игры заключается в уничтожении всех блоков на экране с помощью шарика и платформы, управляемой игроком. 

\textbf{Игровой процесс и цель:} ~\cite{wikiRUHabr7}

Игра "Арканоид" является классическим представителем аркадных игр, где игрок управляет платформой, отбивая мячик таким образом, чтобы разрушить все блоки, расположенные в верхней части экрана. Цель игры — очистить экран от всех блоков, при этом предотвращая падение мяча за нижний край экрана.

\textbf{Игровое поле и элементы:}

Игровое поле обычно представляет собой прямоугольную область, где сверху расположены различные блоки, которые могут иметь разные цвета, формы и свойства. В нижней части экрана находится платформа (обычно это горизонтальная линия), которую управляет игрок, и мяч, который отскакивает от платформы в блоки.

    Платформа Игрок управляет платформой, которая расположена внизу экрана. Платформа используется для отбивания шарика и направления его к блокам.

    Шарик: Шарик начинает своё движение сверху экрана и отскакивает от платформы и стенок. Цель игрока — не дать шарику упасть за пределы экрана.

    Блоки: Экран заполнен различными блоками, которые нужно разрушить. Блоки могут иметь разные цвета, а также различную стойкость к ударам шарика.

    Бонусы: В процессе игры могут появляться различные бонусы, которые улучшают возможности игрока (например, увеличение платформы, ускорение шарика, дополнительные жизни и т.д.) или дают дополнительные очки.

    Уровни: Игра состоит из нескольких уровней, каждый из которых представляет собой новое расположение блоков

 \textbf{Стратегии и тактика игры:}
 
Игра "Арканоид" сочетает в себе реакцию, ловкость и стратегию. Эффективная игра требует от игрока умения точно контролировать платформу для отбивания мяча в нужном направлении и выбора моментов для использования бонусов. Стратегия включает в себя планирование пути мяча для максимального разрушения блоков и предотвращения его потери.

 \textbf{Победа и поражение:}
 
Игрок выигрывает, когда все блоки на экране уничтожены, и проигрывает, если мяч упадет за нижний край экрана (при этом обычно теряется одна из жизней игрока).

Игра "Арканоид" остается популярной благодаря своей простоте, динамике и возможности разработки различных стратегий для достижения максимальных результатов.


\section{\label{sec:ch01/sec02}Python}
    
\subsection{\label{subsec:ch01/sec02/sub01}Теория Python}
Python ~\cite{book1author}представляет собой популярный высокоуровневый язык программирования, который широко используется для создания различных типов приложений, включая веб-приложения ~\cite{book1author4}, настольные программы, игры ~\cite{book1author2}, а также в области машинного обучения и исследований искусственного интеллекта ~\cite{book1author3}.

\textbf{Основные характеристики Python:}

\begin{itemize}

   \item  \textbf{Простота и ясность синтаксиса:} ~\cite{wikiRUHabr6}
   
   Синтаксис Python лаконичен и понятен, что упрощает разработку и поддержку кода, делая его доступным для широкого круга разработчиков.

   \item  \textbf{Бесплатность и открытый исходный код:}
   
   Python распространяется бесплатно и имеет открытый исходный код, что способствует его доступности и распространению среди разработчиков.

   \item  \textbf{Кроссплатформенность:} ~\cite{wikiRUHabr3}
   
   Приложения, написанные на Python, могут быть легко запущены на различных операционных системах без изменений в исходном коде, благодаря своей кроссплатформенной природе.

   \item  \textbf{Расширяемость и интеграция:}
   
   Python легко расширяется за счет использования библиотек, написанных на C или C++, а также может быть использован как язык расширения для настраиваемых приложений.
   
   \item  \textbf{Недостатки Python:} ~\cite{wikiRUHabr2}
   
     Python может быть менее эффективным по скорости выполнения и потреблению памяти по сравнению с компилируемыми языками, такими как C или C++. 
    
\end{itemize}
\subsection{\label{subsec:ch01/sec02/sub02}Библиотека PySide6}
PySide6 ~\cite{book5author} является кроссплатформенной библиотекой для создания графических пользовательских интерфейсов (GUI) на языке программирования Python. Она основана на Qt, что делает её мощным инструментом для разработки современных и функциональных приложений.
\begin{itemize}
\item \textbf{Основные характеристики и возможности PySide6:} ~\cite{wikiRUHabr}

\begin{itemize}
     \item  \textbf{Кроссплатформенность:}PySide6 позволяет разрабатывать приложения, которые могут быть запущены на различных операционных системах, включая Windows, macOS и Linux. 

     \item  \textbf{Широкий набор компонентов интерфейса:} Библиотека предоставляет множество готовых компонентов интерфейса, таких как кнопки, поля ввода, таблицы, меню и другие, что ускоряет процесс создания разнообразных элементов пользовательского интерфейса.

     \item  \textbf{Гибкость и настраиваемость:} PySide6 поддерживает различные стили и темы оформления, а также позволяет расширять функциональность с помощью собственных компонентов и модулей.

     \item  \textbf{Обширная документация и сообщество:} PySide6 обладает хорошей документацией, содержащей примеры кода и учебные материалы, что упрощает изучение и использование библиотеки. Существует активное сообщество разработчиков, готовых помочь друг другу.
\end{itemize}

\item \textbf {Основные модули PySide6:} ~\cite{wikiRUQtDocumentation}
\begin{itemize}
    \item \textbf{QtWidgets:} Основной модуль PySide6, который предоставляет классы для создания и управления стандартными элементами пользовательского интерфейса, такими как кнопки, поля ввода, списки и т.д.

    \item \textbf{QtGui:} ~\cite{wikiRUHabr4} Модуль, предоставляющий классы для работы с графическими элементами интерфейса, включая рисование, обработку событий и работу с изображениями.

    \item \textbf{QtCore:} Включает базовые классы и функции, необходимые для работы с Qt, такие как управление потоками, таймерами, строками и коллекциями данных.
\end{itemize}
PySide6 представляет собой мощный инструмент для разработки современных кроссплатформенных приложений с графическим интерфейсом на языке Python, обеспечивая высокую производительность и удобство использования.
\end{itemize}


