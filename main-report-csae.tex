\documentclass[14pt, oneside]{altsu-report}

\worktype{Отчёт по практике на тему:}
\title{Арканоид}
\author{М.\,С.~Дягилев}
\groupnumber{5.205-2}
\GradebookNumber{1337}
\supervisor{И.\,А.~Шмаков}
\supervisordegree{ст.преп.}
\ministry{Министерство науки и высшего образования}
\country{Российской Федерации}
\fulluniversityname{ФГБОУ ВО Алтайский государственный университет}
\institute{Институт цифровых технологий, электроники и физики}
\department{Кафедра вычислительной техники и электроники}
\departmentchief{В.\,В.~Пашнев}
\departmentchiefdegree{к.ф.-м.н., доцент}
\shortdepartment{ВТиЭ}
\abstractRU{ Объект исследования: Разработка игры на Python с использованием PySide6. Цель работы - создание интерактивной игры с использованием инструментов Python и библиотеки PySide6. Метод исследования - разработка игры с применением объектно-ориентированного программирования, графического интерфейса пользователя (GUI) и игровой логики. В ходе работы были использованы следующие аппаратные и программные средства: персональный компьютер с операционной системой Windows 10, Python 3.9+ и PySide6 6.2.4.

Результатом работы является разработанная интерактивная игра с графическим интерфейсом пользователя. Для создания игровых объектов и логики были применены методы объектно-ориентированного программирования. Также были использованы современные методы Python и PySide6 для создания игры.

Основные конструктивные, технологические и технико-эксплуатационные характеристики работы включают использование языка программирования Python 3.9+, библиотеки GUI PySide6 6.2.4, операционной системы Windows 10 и тип игры - 2D аркада.

Разработанная игра может быть использованав качестве развлечения и демонстрация возможностей Python и PySide6. Она является бесплатным продуктом.
}
\abstractEN{Большой текст на английском!}
\keysRU{Python, PySide6, программирование, GUI}
\keysEN{computer simulation, distributed version control}

\date{\the\year}

% Подключение файлов с библиотекой.
\addbibresource{graduate-students.bib}

% Пакет для отладки отступов.
%\usepackage{showframe}

\begin{document}
\maketitle

\setcounter{page}{2}
\makeabstract
\tableofcontents

\chapter*{Введение}
\phantomsection\addcontentsline{toc}{chapter}{ВВЕДЕНИЕ}

\textbf{Актуальность} 
\begin{enumerate}
Процесс создания игры учит основам Python, ООП, алгоритмам, дизайну и другим важным навыкам программирования. Создание игры с помощью PySide6 позволяет ознакомиться с этой библиотекой и научиться создавать красивые и функциональные GUI на Python.
\end{enumerate}

\textbf{Цель}
\begin{enumerate}
Целю игры является создание игры "Арканоид" на языке Python c использованием библиотеки PySide6.
\end{enumerate}

\textbf{Задачи:}
\begin{enumerate}
\item Реализовать игровые элементы:
\begin{itemize}
    \item Платформа.
    \item Мяч.
    \item Кирпичики.
\end{itemize}
\item Игровой процесс
\begin{itemize}
    \item Обрабатывать столкновение мяча с платформой и кирпичиками.
    \item Изменять направление мяча после столкновения.
    \item Завершать игры после достижении мячом нижней границы.
    \item Реализовать управление платформой с помощью мыши.
\end{itemize}
\item Меню
\begin{itemize}
    \item Кнопка "Играть"
    \item Кнопка "Help"
\end{itemize}
\end{enumerate}

% Подключение первой главы (теория):
\chapter{\label{ch:ch01}ТЕОРИТИЧЕСКАЯ ГЛАВА} % Нужно сделать главу в содержании заглавными буквами

\section{\label{sec:ch01/sec01} Арканоид}

\textbf{Общие сведения об игре "Арканоид"}

Игра Арканоид — классическая аркадная игра, выпущенная в 1986 году компанией Taito.
~\cite{wikiRUArkanoid} Цель игры заключается в уничтожении всех блоков на экране с помощью шарика и платформы, управляемой игроком. 

\textbf{Игровой процесс и цель:} ~\cite{wikiRUHabr7}

Игра "Арканоид" является классическим представителем аркадных игр, где игрок управляет платформой, отбивая мячик таким образом, чтобы разрушить все блоки, расположенные в верхней части экрана. Цель игры — очистить экран от всех блоков, при этом предотвращая падение мяча за нижний край экрана.

\textbf{Игровое поле и элементы:}

Игровое поле обычно представляет собой прямоугольную область, где сверху расположены различные блоки, которые могут иметь разные цвета, формы и свойства. В нижней части экрана находится платформа (обычно это горизонтальная линия), которую управляет игрок, и мяч, который отскакивает от платформы в блоки.

    Платформа Игрок управляет платформой, которая расположена внизу экрана. Платформа используется для отбивания шарика и направления его к блокам.

    Шарик: Шарик начинает своё движение сверху экрана и отскакивает от платформы и стенок. Цель игрока — не дать шарику упасть за пределы экрана.

    Блоки: Экран заполнен различными блоками, которые нужно разрушить. Блоки могут иметь разные цвета, а также различную стойкость к ударам шарика.

    Бонусы: В процессе игры могут появляться различные бонусы, которые улучшают возможности игрока (например, увеличение платформы, ускорение шарика, дополнительные жизни и т.д.) или дают дополнительные очки.

    Уровни: Игра состоит из нескольких уровней, каждый из которых представляет собой новое расположение блоков

 \textbf{Стратегии и тактика игры:}
 
Игра "Арканоид" сочетает в себе реакцию, ловкость и стратегию. Эффективная игра требует от игрока умения точно контролировать платформу для отбивания мяча в нужном направлении и выбора моментов для использования бонусов. Стратегия включает в себя планирование пути мяча для максимального разрушения блоков и предотвращения его потери.

 \textbf{Победа и поражение:}
 
Игрок выигрывает, когда все блоки на экране уничтожены, и проигрывает, если мяч упадет за нижний край экрана (при этом обычно теряется одна из жизней игрока).

Игра "Арканоид" остается популярной благодаря своей простоте, динамике и возможности разработки различных стратегий для достижения максимальных результатов.


\section{\label{sec:ch01/sec02}Раздел 2: Python}
    
\subsection{\label{subsec:ch01/sec02/sub01}Теория Python}
Python ~\cite{book1author}представляет собой популярный высокоуровневый язык программирования, который широко используется для создания различных типов приложений, включая веб-приложения ~\cite{book1author4}, настольные программы, игры ~\cite{book1author2}, а также в области машинного обучения и исследований искусственного интеллекта ~\cite{book1author3}.

\textbf{Основные характеристики Python:}

\begin{itemize}

   \item  \textbf{Простота и ясность синтаксиса:} ~\cite{wikiRUHabr6}
   
   Синтаксис Python лаконичен и понятен, что упрощает разработку и поддержку кода, делая его доступным для широкого круга разработчиков.

   \item  \textbf{Бесплатность и открытый исходный код:}
   
   Python распространяется бесплатно и имеет открытый исходный код, что способствует его доступности и распространению среди разработчиков.

   \item  \textbf{Кроссплатформенность:} ~\cite{wikiRUHabr3}
   
   Приложения, написанные на Python, могут быть легко запущены на различных операционных системах без изменений в исходном коде, благодаря своей кроссплатформенной природе.

   \item  \textbf{Расширяемость и интеграция:}
   
   Python легко расширяется за счет использования библиотек, написанных на C или C++, а также может быть использован как язык расширения для настраиваемых приложений.
   
   \item  \textbf{Недостатки Python:} ~\cite{wikiRUHabr2}
   
     Python может быть менее эффективным по скорости выполнения и потреблению памяти по сравнению с компилируемыми языками, такими как C или C++. 
    
\end{itemize}
\subsection{\label{subsec:ch01/sec02/sub02}Библиотека PySide6}
PySide6 ~\cite{book5author} является кроссплатформенной библиотекой для создания графических пользовательских интерфейсов (GUI) на языке программирования Python. Она основана на Qt, что делает её мощным инструментом для разработки современных и функциональных приложений.
\begin{itemize}
\item \textbf{Основные характеристики и возможности PySide6:} ~\cite{wikiRUHabr}

\begin{itemize}
     \item  \textbf{Кроссплатформенность:}PySide6 позволяет разрабатывать приложения, которые могут быть запущены на различных операционных системах, включая Windows, macOS и Linux. 

     \item  \textbf{Широкий набор компонентов интерфейса:} Библиотека предоставляет множество готовых компонентов интерфейса, таких как кнопки, поля ввода, таблицы, меню и другие, что ускоряет процесс создания разнообразных элементов пользовательского интерфейса.

     \item  \textbf{Гибкость и настраиваемость:} PySide6 поддерживает различные стили и темы оформления, а также позволяет расширять функциональность с помощью собственных компонентов и модулей.

     \item  \textbf{Обширная документация и сообщество:} PySide6 обладает хорошей документацией, содержащей примеры кода и учебные материалы, что упрощает изучение и использование библиотеки. Существует активное сообщество разработчиков, готовых помочь друг другу.
\end{itemize}

\item \textbf {Основные модули PySide6:} ~\cite{wikiRUQtDocumentation}
\begin{itemize}
    \item \textbf{QtWidgets:} Основной модуль PySide6, который предоставляет классы для создания и управления стандартными элементами пользовательского интерфейса, такими как кнопки, поля ввода, списки и т.д.

    \item \textbf{QtGui:} ~\cite{wikiRUHabr4} Модуль, предоставляющий классы для работы с графическими элементами интерфейса, включая рисование, обработку событий и работу с изображениями.

    \item \textbf{QtCore:} Включает базовые классы и функции, необходимые для работы с Qt, такие как управление потоками, таймерами, строками и коллекциями данных.
\end{itemize}
PySide6 представляет собой мощный инструмент для разработки современных кроссплатформенных приложений с графическим интерфейсом на языке Python, обеспечивая высокую производительность и удобство использования.
\end{itemize}



% Подключение второй главы (практическая часть):
\chapter{\label{ch:ch02}ГЛАВА 2}

\section{\label{sec:ch02/sec01}Реализация игры}

\subsection{\label{subsec:ch02/sec01/sub01}Структура игры}


\subsection{\label{subsec:ch02/sec01/sub02}Описание основных классов}

\section{\label{sec:ch02/sec02}Графический интерфейс}

\subsection{\label{subsec:ch02/sec02/sub01}Описание внешнего вида программы}

\subsection{\label{subsec:ch02/sec02/sub02}Подраздел 2}


% Подключение третий главы (практическая часть с тестированием):
\chapter{\label{ch:ch03}ГЛАВА 3}

Пример ссылок:
\begin{enumerate}
\item на главу~\ref{ch:ch01};
\item на раздел~\ref{sec:ch01/sec01} главы~\ref{ch:ch01};
\item на раздел~\ref{sec:ch02/sec01} главы~\ref{ch:ch02};
\item на приложение на странице~\pageref{appendix1};
\item на код на странице~\pageref{code:pi-example}.
\end{enumerate}

\section{\label{sec:ch03/sec01}Раздел 1}

\subsection{\label{subsec:ch03/sec01/sub01}Подраздел 1}

\subsection{\label{subsec:ch03/sec01/sub02}Подраздел 2}

\section{\label{sec:ch03/sec02}Раздел 2}

\subsection{\label{subsec:ch03/sec02/sub01}Подраздел 1}

\subsection{\label{subsec:ch03/sec02/sub02}Подраздел 2}




\chapter*{Заключение}
\phantomsection\addcontentsline{toc}{chapter}{ЗАКЛЮЧЕНИЕ}

\begin{enumerate}
\item Пример ссылки на электронный источник~\cite{wikiRUArkanoid,wikiRUQtDocumentation,wikiRUHabr}.
\item Пример ссылки на книгу одного автора~\cite{book1author}.
\item Пример ссылки на книгу 5-ти и более авторов~\cite{book5author}.
\end{enumerate}

\newpage
\phantomsection\addcontentsline{toc}{chapter}{СПИСОК ИСПОЛЬЗОВАННОЙ ЛИТЕРАТУРЫ}
\printbibliography[title={Список использованной литературы}]

\appendix
\newpage
\chapter*{\raggedleft\label{appendix1}Приложение}
\phantomsection\addcontentsline{toc}{chapter}{ПРИЛОЖЕНИЕ}
%\section*{\centering\label{code:appendix}Текст программы}

\begin{center}
\label{code:appendix}Текст программы
\end{center}

\begin{code}
\captionof*{listing}{\centering\label{code:pi-example}Пример программы вычисления числа $\pi$ на языке \textit{C} с использованием \textit{MPI} (пример из https://ru.wikipedia
.org/wiki/Message\_Passing\_Interface)}
\vspace{-1cm}\inputminted{C}{src/pi-mpi.c}
\end{code}

\end{document}

