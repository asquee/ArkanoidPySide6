\chapter{\label{ch:ch01}ГЛАВА 1} % Нужно сделать главу в содержании заглавными буквами

\section{\label{sec:ch01/sec01} Арканоид}

\subsection{\label{subsec:ch01/sec01/sub01}Об игре}
***что такое "Арканоид" и какие правила игры.***
Пример <<ковычек>> и тире ---.

Пример нумерованного списка:
\begin{enumerate}
\item Первый элемент.
\item Второй элемент.
\end{enumerate}

\subsection{\label{subsec:ch01/sec01/sub02}Цель игры и основные механики}
***Объяснить цель игры и основные механики.***
Пример маркерованного списка:
\begin{itemize}
\item первый элемент;
\item второй элемент.
\end{itemize}

\section{\label{sec:ch01/sec02}Раздел 2: Python}
    
\subsection{\label{subsec:ch01/sec02/sub01}Теория Python}
***Рассказать о языке питон, почему был выбран именно он и т.д***
Пример вложенного нумерованного списка:
\begin{enumerate}
\item Первый элемент:
\begin{enumerate}
\item Первый элемент первого элемента;
\item Второй элемент первого элемента;
\end{enumerate}
\item Второй элемент:
\begin{enumerate}
\item Первый элемент второго элемента;
\item Второй элемент второго элемента.
\end{enumerate}
\end{enumerate}

\subsection{\label{subsec:ch01/sec02/sub02}PySide}
*** описать, что такое PySide6 и для чего она используется.***
***Упомянуть основные возможности и преимущества использования PySide6.***
Пример вложенного маркерованного списка:
\begin{itemize}
\item первый элемент:
\begin{itemize}
\item первый элемент первого элемента;
\item второй элемент первого элемента;
\end{itemize}
\item Второй элемент:
\begin{itemize}
\item первый элемент второго элемента;
\item второй элемент второго элемента.
\end{itemize}
\end{itemize}

